% ************************** Thesis Abstract *****************************
% Use `abstract' as an option in the document class to print only the titlepage and the abstract.
\begin{abstract}
Humans convey their thoughts, emotions, and intentions through a concert of social displays: voice, facial expressions, hand gestures, and body posture, collectively referred to as social signals. Despite advances in machine perception, machines are unable to discern the subtle and momentary nuances that carry so much of the information and context of human communication. The encoding of conveyed information by social signals, particularly in nonverbal communication, is still poorly understood, and thus it is unclear how to teach machines to use such social signals to make them collaborative partners rather than tools that we use. A major obstacle to scientific progress in this direction is the inability to sense and measure the broad spectrum of behavioral cues in groups of interacting individuals, which hinders applying computational methods to model and understand social signals.

In this thesis, we explore new approaches in sensing, measuring, and modeling social signals to ultimately endow machines with the ability to interpret nonverbal communication. This thesis starts by describing our exploration in building a massively multiview sensor system, the \emph{Panoptic Studio}, that can capture a broad spectrum of human social signaling---including voice, social formations, facial expressions, hand gestures, and body postures---among groups of multiple people. Second, leveraging this system equipped with more than 500 synchronized cameras, we then present a method to measure the subtle 3D movements of anatomical keypoints in face-to-face interaction, providing a new opportunity to computationally study social signals. In the last part of this thesis, we present a social signal prediction task to model nonverbal communication in a data-driven manner.  We establish a new large-scale corpus from hundreds of participants containing various channels of social signal measurements. Leveraging this dataset, we verify that the social signals are predictive each other with strong correlations.
%Humans convey their thoughts, emotions, and intentions through a concert of social displays: voice, facial expressions, hand gestures, and body posture, collectively referred to as social signals. Despite advances in machine perception technology, machines are unable to discern the subtle and momentary nuances that carry so much of the information and context of human communication. The encoding of conveyed information by social signals, particularly in nonverbal communication,  is still poorly understood, and thus it is unclear how to teach machines to use such social signals to make them as collaborative partners of humans rather than a tool. A major obstacle to scientific progress in this direction is the inability to sense and measure the wide spectrum of behavioral cues in groups of interacting individuals, which hinders applying computational methods to model and understand social signals.
%
%In this thesis, we explore new approaches in sensing, measuring, and modeling social signals to ultimately endow machines with nonverbal communication abilities. The core idea of this thesis is to enable machines to automatically learn such social skills from the data, where the data is collected by sensing and measuring the various behavioral cues that naturally emerge during social interactions. This thesis starts by describing our exploration in building a massively multiview sensor system,  the \emph{Panoptic Studio}, that can capture the wide spectrum of human social signaling---including voice, social formations, facial expressions, hand gestures, and body postures---among groups of multiple people. Leveraging more than 500 synchronized cameras, we then present a method to markerlessly measure the subtle 3D movements of anatomical keypoints in face-to-face interactions, providing a new opportunity to computationally study social signals. In the last part of this thesis, we present a social signal prediction task to model nonverbal communication in a data-driven manner.  We collect a new large-scale corpus from hundreds of participants containing various channels of social signal measurements. Leveraging this dataset, we verify that the social signals are predictive each other with strong correlations. Finally, we present a \emph{social Artificial Intelligence} equipped with nonverbal communications skills by compositing our social signal prediction models.
\end{abstract}





%\begin{abstract}
%Humans use subtle and elaborate body signals to convey their thoughts, emotions, and intentions. ``Kinesics" is a term that refers to the study of such body movements used in social communication, including facial expressions and hand gestures. Understanding kinesic signals is fundamental to understanding human communication; it is among the key technical barriers to making machines that can genuinely communicate with humans. Yet, the encoding of conveyed information by body movement is still poorly understood.
%
%This thesis proposal is focused on two major challenges in building a computational understanding of kinesic communication: (1) measuring full body motion as a continuous high bandwidth kinesic signal; and (2) modeling kinesic communication as information flow between coupled agents that continuously predict each others' response signals.  To measure kinesic signals between multiple interacting people, we first develop the \emph{Panoptic Studio}, a massively multiview system composed of more than five hundred camera sensors. The large number of views allows us to develop a method to robustly measure subtle 3D motions of bodies, hands, and faces of all individuals in a large group of people. To this end, a dataset containing 3D kinesic signals of more than two hundred sequences from hundreds of participants is collected and shared publicly.
%
%The Panoptic studio allows us to measure kinesic signals of a large group of interacting people for the first time. We propose to model these signals as information flow in a communication system. The core thesis of our approach is that a meaningful encoding of body movement will emerge from representations that are optimized for efficient prediction of these kinesic communication signals. We hope to see this approach inspire continuous and quantitative models in the future study of social behavior. 
%
%%approach to model kinesic signals based on the large scale dataset we have collected. In our framework, our model takes social signals from other people as input and predicts the signals of the target person as output. Our goal is to build a model to mimic human's nonverbal interactions by understanding and producing kinesics signals. To this end, we also propose a way to quantify influence of each body part in social communication.
%
%\end{abstract}
