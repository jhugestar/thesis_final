% ************************** Thesis Abstract *****************************
% Use `abstract' as an option in the document class to print only the titlepage and the abstract.
\begin{abstract}
Humans use subtle and elaborate body signals to convey their thoughts, emotions, and intentions. ``Kinesics" is a term that refers to the study of such body movements used in social communication, including facial expressions and hand gestures. Understanding kinesic signals is fundamental to understanding human communication; it is among the key technical barriers to making machines that can genuinely communicate with humans. Yet, the encoding of conveyed information by body movement is still poorly understood.

This thesis proposal is focused on two major challenges in building a computational understanding of kinesic communication: (1) measuring full body motion as a continuous high bandwidth kinesic signal; and (2) modeling kinesic communication as information flow between coupled agents that continuously predict each others' response signals.  To measure kinesic signals between multiple interacting people, we first develop the \emph{Panoptic Studio}, a massively multiview system composed of more than five hundred camera sensors. The large number of views allows us to develop a method to robustly measure subtle 3D motions of bodies, hands, and faces of all individuals in a large group of people. To this end, a dataset containing 3D kinesic signals of more than two hundred sequences from hundreds of participants is collected and shared publicly.

The Panoptic studio allows us to measure kinesic signals of a large group of interacting people for the first time. We propose to model these signals as information flow in a communication system. The core thesis of our approach is that a meaningful encoding of body movement will emerge from representations that are optimized for efficient prediction of these kinesic communication signals. We hope to see this approach inspire continuous and quantitative models in the future study of social behavior. 

%approach to model kinesic signals based on the large scale dataset we have collected. In our framework, our model takes social signals from other people as input and predicts the signals of the target person as output. Our goal is to build a model to mimic human's nonverbal interactions by understanding and producing kinesics signals. To this end, we also propose a way to quantify influence of each body part in social communication.

\end{abstract}
